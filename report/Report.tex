\documentclass[12pt,a4paper]{report}
\usepackage[utf8]{inputenc}
\usepackage{amsfonts}
\usepackage{setspace}
\usepackage{graphicx}
\usepackage{array}
\usepackage{fancyhdr}
\usepackage{geometry}
\usepackage{ragged2e}
\usepackage{color}
\usepackage{biblatex}
\usepackage{tabularx}


\addbibresource{reference.bib}

\geometry{
a4paper,
total={210mm,297mm},
left=1.15in,
right=0.85in,
top=1.0in,
bottom=1.0in,
}

\begin{document}

\pagestyle{empty}

%%%%%%%%%%%%%%%%%%% Front Page  %%%%%%%%%%%%%%%%%%%%%%%
\begin{center}
{\large \textbf{Visvesvaraya Technological University, Belagavi – 590018}}
\begin{figure}[hbtp]
\centering
\includegraphics[width=2.3cm,height=3cm]{./pic/vtu}
\end{figure}

\textbf{MINI-PROJECT REPORT}
\par
\textbf{ON}
\par
\vspace{6pt}
{\Large \textbf{B-LOCK: A Blockchain based e-vault application}}
\par
\vspace{12pt}
\par
\textit{\textbf{Submitted in partial fulfillment for the award of degree of}}
\par
\vspace{12pt}
\large \textbf{BACHELOR OF ENGINEERING }
\par
\textbf{in}
\par
\large \textbf{COMPUTER SCIENCE \& ENGINEERING}
\par
\vspace{12pt}
\textit{\textbf{Submitted by}}

\begin{center}
\begin{tabular}{l@{\hspace{2cm}}r}
\textbf{\large Abhik L Salian } & \textbf{4SO21CS004} \\
\textbf{\large Ashwitha Shetty} & \textbf{4SO21CS030} \\
\textbf{\large H Karthik P Nayak } & \textbf{4SO21CS058} \\
\textbf{\large Jessica Lillian Mathew } & \textbf{4SO21CS066} \\
\end{tabular}
\end{center}

\vspace{12pt}
\textit{\textbf{Under the Guidance of}}
\par
\vspace{6pt}
\textbf{Dr. Santhosh Kumar DK}
\par
\vspace{2pt}
\normalsize {Associate Professor, Department of CSE}
\par
\begin{figure}[hbtp]
\centering
\includegraphics[scale=0.6]{./pic/sjeclogo}
\end{figure}
\large \textbf{DEPT. OF COMPUTER SCIENCE AND ENGINEERING}
\par \Large \textbf{ST JOSEPH ENGINEERING COLLEGE}
\par 
\textbf{An Autonomous Institution}
\par
{\large{(Affiliated to VTU Belagavi, Recognized by AICTE, Accredited by NBA)}}
\par
{\large \textbf{Vamanjoor, Mangaluru - 575028, Karnataka}}
\par 
{\Large \textbf{2023-24}}
\end{center}
\newpage

%%%%%%%%%%%%%%%%%%%%%%%%%% Certificate Page %%%%%%%%%%%%%%%%%%%%%%%%%%%%%%
\begin{center}
\LARGE \textbf{ST JOSEPH ENGINEERING COLLEGE}
\par
\Large \textbf{An Autonomous Institution}
\par \large{(Affiliated to VTU Belagavi, Recognized by AICTE, Accredited by NBA)}
\par \vspace{3pt}
\large \textbf{Vamanjoor, Mangaluru - 575028, Karnataka}
\par \vspace{12pt}  
\par
\large \textbf{DEPT. OF COMPUTER SCIENCE AND ENGINEERING}
\par
\begin{figure}[hbtp]
\centering
\includegraphics[scale=0.5]{./pic/sjeclogo}
\end{figure}


{\Large \textbf{CERTIFICATE}}
\end{center}
\justifying
\par
\setstretch{1.2}
\vspace{0.10in}
\noindent 
Certified that the Mini-project work entitled \textbf{``B-LOCK: A Blockchain based e-vault application''} carried out by\vspace{2pt} 
\par
\noindent 
\begin{center}
\begin{tabular}{l@{\hspace{2cm}}r}
\textbf{\large Abhik L Salian } & \textbf{4SO21CS004} \\
\textbf{\large Ashwitha Shetty} & \textbf{4SO21CS030} \\
\textbf{\large H Karthik P Nayak } & \textbf{4SO21CS058} \\
\textbf{\large Jessica Lillian Mathew } & \textbf{4SO21CS066} \\
\end{tabular}
\end{center}
\noindent
the bonafide students of VI semester Computer Science \& Engineering in partial fulfillment for the award of Bachelor of Engineering in Computer Science and Engineering of the Visvesvaraya Technological University, Belagavi during the year 2023-2024. It is certified that all suggestions indicated during internal assessment have been incorporated in the report. The project report has been approved as it satisfies the academic requirements in respect of project work prescribed for the said degree. 


\vspace{0.55in}
\par
\vspace{0.65in}
\setstretch{1.15}

\begin{tabularx}{0.95 \textwidth} { 
   >{\raggedright\arraybackslash}X 
   >{\centering\arraybackslash}X 
   >{\raggedleft\arraybackslash}X  }
     \textbf{Ms. L Hema} &  \textbf{Dr Sridevi Saralaya}\\
     Mini-Project Coordinator &   HOD-CSE \\
\end{tabularx}




%%%%%%%%%%%%%%%%%%%%%%%%%% Abstract %%%%%%%%%%%%%%%

\pagestyle{plain}
\setstretch{1.5}
\pagenumbering{roman}
\chapter*{Abstract}
\addcontentsline{toc}{chapter}{\numberline{}Abstract}
The advent of blockchain technology has revolutionized the 
way data is stored and managed, security, transparency, and 
immutability. This report presents the development of a 
blockchain-based eVault, a decentralized 
application designed to securely store, retrieve, download, 
and delete files and documents. The eVault leverages blockchain's 
distributed ledger technology to ensure the integrity and 
confidentiality of users' data.

The primary objective of this project is to provide a robust 
and user-friendly platform for secure document management, 
addressing the increasing demand for data privacy and protection. 
The eVault application allows users to upload files directly 
onto the blockchain, where each document is encrypted and 
assigned a unique identifier for easy retrieval. Users can 
access their documents at any time, with the assurance that 
the data remains untampered due to the immutable nature of 
blockchain records.
 
Key features include secure file upload, efficient retrieval 
using unique identifiers, and comprehensive access control 
mechanisms that ensure data is accessible only to authorized 
users. Performance metrics indicate that the system achieves 
optimal transaction speeds and data retrieval times, even 
under high loads.

Throughout the development process, challenges such as 
scalability and integration with existing systems were 
addressed, resulting in a versatile solution that meets industry 
demands. 

This blockchain-based eVault demonstrates significant potential 
across various industries, offering a secure and efficient 
solution for managing sensitive documents. By addressing current 
limitations in data storage and retrieval, this project paves 
the way for innovative advancements in digital security and 
privacy.

\setstretch{1.2}
\renewcommand{\contentsname}{Table of Contents}
\tableofcontents
\addcontentsline{toc}{chapter}{\numberline{}Table of Contents}
\listoffigures
\addcontentsline{toc}{chapter}{\numberline{}List of Figures}
\listoftables
\addcontentsline{toc}{chapter}{\numberline{}List of Tables}
\newpage

%%%%%%%%%%%%%%%%%%%%% Headers and Footers %%%%%%%%%%%%%%%

\pagestyle{fancy}
\fancyhf{}
\lhead{\fontsize{10}{12} \selectfont Video Digest: A Query Based Dynamic Video Synopsis System}
\rhead{\fontsize{10}{12} \selectfont Chapter \thechapter}
\lfoot{\fontsize{10}{12} \selectfont Department of Computer Science and Engineering, SJEC, Mangaluru}
\rfoot{\fontsize{10}{12} \selectfont Page \thepage}
\renewcommand{\headrulewidth}{0.5pt}
\renewcommand{\footrulewidth}{0.5pt}


%%%%%%%%%%%%%%%%%%%%%%% CHapetr 1 Introduction %%%%%%%%%%%%%

\setstretch{1.2}
\pagenumbering{arabic}

\chapter{Introduction}

\section{Background}

In today's digital age, the management and security of data have become paramount across numerous industries. With the rapid increase in data generation, there is a growing need for efficient and secure systems to store, retrieve, and manage sensitive information. Traditional storage solutions often struggle with issues of security breaches, data tampering, and unauthorized access. 

Blockchain technology has emerged as a revolutionary solution to these challenges, offering a decentralized and immutable ledger that enhances data integrity, transparency, and security. By leveraging blockchain, we can ensure that data is stored in a manner that is resistant to tampering and unauthorized access, making it an ideal solution for managing sensitive documents and files.

This report introduces the development of a blockchain-based eVault, a decentralized application designed to provide a secure platform for storing, retrieving, downloading, and deleting files and documents. The eVault utilizes blockchain's unique capabilities to ensure that user data is encrypted, protected, and accessible only to authorized users, addressing the critical need for data privacy and security in various sectors.

The primary goal of this project is to create a user-friendly platform that leverages blockchain technology to offer a robust solution for document management. This platform allows users to securely upload files to the blockchain, ensuring that each document is encrypted and assigned a unique identifier for easy and secure retrieval. The immutable nature of blockchain ensures that once data is stored, it cannot be altered or deleted without proper authorization, providing a high level of trust and reliability.


\section{Problem statement }
In today's digital age, securing sensitive data is a major challenge due to the limitations of traditional storage systems, which often struggle with data breaches, unauthorized access, and tampering. There is an urgent need for a reliable solution that ensures data integrity, privacy, and security. The blockchain-based eVault addresses these issues by providing a decentralized platform for secure file storage, retrieval, download, and deletion, leveraging blockchain technology to guarantee data protection and user control.

\section{Scope}
The blockchain-based eVault project aims to develop a secure and efficient platform for managing sensitive documents across various industries. This project focuses on creating a user-friendly application that allows users to store, access, and manage files with complete confidence in their security.

The eVault will feature encrypted file storage, ensuring that all documents are tamper-proof and accessible only to authorized users. It will implement unique identifiers and access control mechanisms for efficient data retrieval and user management, providing a reliable alternative to conventional storage systems.

The application is particularly suited for sectors where data privacy is critical, such as finance, healthcare, and legal services. By offering a robust solution for secure document management, the eVault project addresses the growing demand for data security and privacy, paving the way for innovative advancements in digital information management.

%---------------------------- Chapter TWO --------------------------

\chapter{Software Requirements Specification}

\section{Introduction}

The purpose of this Software Requirements Specification (SRS) document is to outline the functional and non-functional requirements for the blockchain-based eVault project. The eVault is a decentralized application designed to provide a secure and user-friendly platform for storing, retrieving, downloading, and deleting files using blockchain technology. This document will serve as a guide for the development team and stakeholders, ensuring that all aspects of the system are well-defined and aligned with the project's goals.

This document is structured into two main sections: Functional Requirements and Non-Functional Requirements. The Functional Requirements section describes the core functionalities that the eVault must provide to meet the user's needs. The Non-Functional Requirements section details the quality attributes, performance standards, and other constraints that the system must adhere to for optimal operation.

\section{Functional Requirements}

The functional requirements define the specific behavior and functionalities of the eVault system. These requirements ensure that the application meets user expectations and provides the necessary capabilities for secure document management. The key functional requirements for the eVault include:

\textbf{1. User Registration and Authentication:}
   - The system must allow users to register and create an account using a secure registration process.
   - Users must be able to log in using their credentials, with multi-factor authentication for added security.
   - Passwords must be stored securely using encryption.

\textbf{2. Secure File Upload:}
   - Users must be able to upload files to the eVault, where each file is encrypted before storage.
   - The system must assign a unique identifier to each file for easy retrieval and management.

\textbf{3. File Retrieval and Access Control:}
   - Users must be able to search and retrieve files using the unique identifiers.
   - Access control mechanisms must be implemented to ensure that only authorized users can access specific files.

\textbf{4. File Download and Deletion:}
   - Users must be able to download files securely, with the option to decrypt them upon download.
   - Users should be able to delete files, ensuring that the deletion process is secure and irreversible.

\textbf{5. Blockchain Integration:}
   - The system must integrate with a blockchain platform to ensure the immutability and security of stored files.
   - All transactions related to file operations (upload, download, delete) must be recorded on the blockchain for auditability.

\textbf{6. User Interface:}
   - The application must provide a user-friendly interface that is intuitive and easy to navigate.
   - The interface should support multiple languages and accessibility features.

\section{Non-Functional Requirements}

The non-functional requirements define the quality attributes, performance standards, and other constraints of the eVault system. These requirements ensure that the application operates efficiently and meets user expectations regarding performance, security, and usability. The key non-functional requirements for the eVault include:

\textbf{1. Performance:}
   - The system must handle a high volume of transactions and file uploads/downloads without significant delays.
   - File retrieval times should not exceed a specified threshold, ensuring quick access to documents.

\textbf{2. Scalability:}
   - The eVault must be able to scale to accommodate a growing number of users and files.
   - The system should support horizontal scaling to handle increased demand.

\textbf{3. Security:}
   - All data, including user credentials and files, must be encrypted to protect against unauthorized access and breaches.
   - The system should comply with industry security standards and regulations (e.g., GDPR, HIPAA).

\textbf{4. Reliability:}
   - The eVault must provide high availability and uptime, minimizing downtime and disruptions.
   - The system should have backup and recovery mechanisms to ensure data integrity and availability.

\textbf{5. Usability:}
   - The application should be intuitive and easy to use, with clear instructions and help resources available.
   - User feedback mechanisms should be in place to gather input for future improvements.

\textbf{6. Compatibility:}
   - The eVault must be compatible with various operating systems and devices, including desktops, tablets, and smartphones.
   - The system should support major web browsers and adhere to web standards.

\textbf{7. Maintainability:}
   - The system architecture should be modular and easy to maintain, allowing for future updates and enhancements.
   - Code documentation and best practices should be followed to facilitate development and troubleshooting.

   \section{User Interface Requirements}

   This section outlines the design and functional requirements of the user interface (UI) for the eVault system. It includes considerations for usability, accessibility, and aesthetic appeal, ensuring an intuitive experience for all users.
   
   \subsection{Design Principles}
   The user interface should adhere to modern design principles to enhance usability and user satisfaction.
   \begin{itemize}
       \item \textbf{Consistency:} The UI should maintain consistency in design elements across all screens, including fonts, colors, and button styles.
       \item \textbf{Simplicity:} The design should be clean and clutter-free, emphasizing essential functions to avoid overwhelming the user.
       \item \textbf{Responsiveness:} The interface must be responsive and adaptable to various screen sizes and devices, including desktops, tablets, and smartphones.
   \end{itemize}
   
   \subsection{Accessibility}
   The application should be accessible to all users, including those with disabilities, by adhering to accessibility standards.
   \begin{itemize}
       \item \textbf{Screen Readers:} The UI should support screen readers to assist visually impaired users in navigating the application.
       \item \textbf{Keyboard Navigation:} Users should be able to navigate the application using keyboard shortcuts without relying on a mouse.
       \item \textbf{Color Contrast:} Ensure sufficient contrast between text and background colors to improve readability for users with visual impairments.
   \end{itemize}
   
   \subsection{User Feedback}
   Feedback mechanisms should be implemented to gather user input and improve the application over time.
   \begin{itemize}
       \item \textbf{Error Messages:} Provide clear and informative error messages to guide users in resolving issues.
       \item \textbf{User Surveys:} Periodic surveys can be conducted to collect user feedback on the UI and identify areas for improvement.
       \item \textbf{Help Resources:} Offer tutorials and help sections to assist users in understanding and using the application's features.
   \end{itemize}
   
   \section{Software Requirements}
   
   This section specifies the technical requirements of the eVault system, covering hardware, software, network, and operational aspects to ensure efficient and reliable performance.
   
   \subsection{Hardware Requirements}
   The eVault system requires specific hardware configurations to operate efficiently and support user needs.
   \begin{itemize}
       \item \textbf{Server Specifications:} Minimum server requirements include 16 GB RAM, 4-core CPU, and 500 GB SSD for optimal performance.
       \item \textbf{User Devices:} Users should have access to devices with at least 4 GB RAM and a modern web browser to use the application effectively.
   \end{itemize}
   
   \subsection{Software Dependencies}
   The eVault relies on various software components and libraries to function correctly.
   \begin{itemize}
       \item \textbf{Operating Systems:} Compatible with Windows, macOS, and Linux for server deployments.
       \item \textbf{Database Systems:} Supports MySQL and MongoDB for data storage and management.
       \item \textbf{Programming Languages:} Developed using Python and JavaScript, leveraging frameworks such as Django and React.
   \end{itemize}
   
   \subsection{Network Requirements}
   Ensure stable and secure network connectivity to support application operations and data transmission.
   \begin{itemize}
       \item \textbf{Bandwidth:} The application requires a minimum internet speed of 10 Mbps for smooth operation.
       \item \textbf{Firewall Settings:} Configure firewalls to allow traffic through necessary ports while blocking unauthorized access.
       \item \textbf{Data Encryption:} All data transmitted over the network must be encrypted using TLS to prevent interception.
   \end{itemize}
   



\chapter{System Design}
paragraph contents... 
\section{Architecture Design}
\begin{figure}[hbtp]
\centering
\includegraphics[width=4in,height=3in]{./pic/sjeclogo.png}
\caption{System Architecture Diagram}
\end{figure}
This Figure 5.1 illustrates a high-level overview of the audio visual speech separation system. It is important to note that the specific techniques, algorithms, and models used in each component can vary depending on the implementation approach and the requirements of the system.
\section{Decomposition Description}
\begin{figure}[hbtp]
\centering
\includegraphics[width=5in,height=3in]{./pic/sjeclogo.png}
\caption{Flow chart}
\end{figure}
\newpage
Figure 5.2 represent the flow chart of the proposed system. In audio visual speech separation, the goal is to decompose an audio signal containing multiple overlapping speakers into individual speech signals corresponding to each speaker. The decomposition process involves separating the desired speech signals from the background noise and other interfering sounds.

\section{Data Flow Design}
\par
The audio input undergoes pre-processing, while the visual input is processed to extract relevant cues. The pre-processed audio and processed visual data are then integrated. From the integrated representation, features are extracted. These features are utilized in the speech separation stage, where individual speech signals are separated from the mixture. Post-processing techniques are applied to enhance the quality of the separated speech signals. Finally, the individual speech signals are outputted as the result of the system. The data flow design ensures a sequential flow of operations, starting from capturing and processing the inputs, integrating the audio-visual information, extracting features, performing speech separation, applying post-processing, and generating the output. This design allows for effective processing and separation of audio visual data to obtain distinct speech signals from overlapping speakers.
\\
\begin{figure}[hbtp]
\centering
\includegraphics[scale=0.8]{./pic/sample.jpg}
\caption{Dataflow design}
\end{figure}




\chapter{Implementation}
the chapter contains paragraph contents.(Pseudocode, Algorithm etc), for ex: Check the following data
\section{Audio Extraction}
\par
Audio extraction is the process of isolating and extracting the audio content from a multimedia source, such as a video file. It involves separating the audio track from the accompanying video or other elements to obtain a standalone audio file representing the sound present in the source material.

\begin{figure}[hbtp]
\centering
\includegraphics[width=5in,height=3in]{./pic/sjeclogo.png}
\caption{code snippet for audio extraction}
\end{figure}

\section{Speech Separation}
\par SpeechBrain is an open-source framework 

\subsection{Sepformer}
\par SepFormer is an algorithm for speech separation that utilizes self-attention mechanisms. It employs a transformer-based architecture to capture long-range dependencies and model the relationships between time-frequency points in the audio mixture, enabling the separation of multiple speech sources from the mixture.


\begin{figure}[hbtp]
\centering
\includegraphics[width=5in,height=3in]{./pic/sjeclogo.png}
\caption{code snippet for speech separation}
\end{figure}

\section{Speech Enhancement}
\subsection{Lite Audio Visual Speech Enhancement}
\par
Lite AVSE algorithm is used for the separation and enhancement of the speech. The system 
includes two visual data compression techniques and removes the visual feature extraction 
network from the training model, yielding better online computation efficiency. As for the audio 
features, short-time Fourier transform (STFT) is calculated of 3-second audio segments. Each 
time-frequency (TF) bin contains the real and imaginary parts of a complex number, both of 
which used as input. Power-law compression used to prevent loud audio from overwhelming soft 
audio. The same processing is applied to both the noisy signal and the clean reference signal.

\begin{figure}[hbtp]
\centering
\includegraphics[width=5in,height=3in]{./pic/sample.jpg}
\caption{code snippet for speech enhancement using LAVSE}
\end{figure}

\subsection{Spectral Subtraction}
Spectral subtraction is a technique used in audio signal processing to reduce background noise from an audio signal. It involves estimating the noise spectrum from a noisy signal and subtracting it from the noisy spectrum to enhance the desired signal. The resulting spectrum is then transformed back into the time domain to obtain a cleaner audio signal.
\newpage
\begin{figure}[hbtp]
\centering
\includegraphics[width=5in,height=3in]{./pic/sample.jpg}
\caption{code snippet for speech enhancement using spectral subtraction}
\end{figure}

\section{Speaker Detection}
\par The cv2 functions provide methods to load the pre-trained models, apply them to images or video frames, and draw bounding boxes around the detected faces. By leveraging cv2's face detection capabilities, you can automate tasks such as facial recognition, emotion analysis, or face tracking in various applications like surveillance, biometrics, or augmented reality.


\begin{figure}[hbtp]
\centering
\includegraphics[width=5in,height=3in]{./pic/sample.jpg}
\caption{code snippet for speaker detection}
\end{figure}







\chapter{Results and Discussion}
this chapter contains the paragraphs as shown below
\section{Face detection}
\begin{figure} [hbtp]
\centering
\includegraphics[width=5in,height=3in]{./pic/sample.jpg}
\caption{Face detection}
\end{figure}
\par Above figure 8.1 shows initial face detection process using opencv and dlib. It convert the image to grayscale, apply the model using cv2.detectMultiScale(), and draw bounding boxes around the detected faces using cv2.rectangle(). Display or save the result using cv2.imshow() or cv2.imwrite().

\section{Speaker recognition}
\begin{figure} [hbtp]
\centering
\includegraphics[width=5in,height=3in]{./pic/sample.jpg}
\caption{Speaker recognition 1,person 1}
\end{figure}

\begin{figure} [hbtp]
\centering
\includegraphics[width=5in,height=3in]{./pic/sample.jpg}
\caption{Speaker recognition 1,person 2}
\end{figure}

\begin{figure} [hbtp]
\centering
\includegraphics[width=5in,height=3in]{./pic/sample.jpg}
\caption{Speaker recognition 2,person 1}
\end{figure}

\begin{figure} [hbtp]
\centering
\includegraphics[width=5in,height=3in]{./pic/sample.jpg}
\caption{Speaker recognition 2,person 2}
\end{figure}

\begin{figure} [hbtp]
\centering
\includegraphics[width=5in,height=3in]{./pic/sample.jpg}
\caption{Speaker recognition 3,person 1}
\end{figure}

\begin{figure} [hbtp]
\centering
\includegraphics[width=5in,height=3in]{./pic/sample.jpg}
\caption{Speaker recognition 3,person 2}
\end{figure}
\newpage
\par Above figures from 8.2 to 8.7 shows speaker recognition process using opencv and dlib.Speaker detection using cv2 and dlib involves utilizing dlib's pre-trained models along with cv2 functions to detect and locate human faces. By combining face detection with additional techniques such as audio analysis or lip movement tracking, speaker detection can be achieved in various applications like video conferencing or surveillance.

\chapter{Conclusion and Future Work}
\cite{bashir2021subjective} \cite{mittal2016}
The Project will help in narrowing the imprecise communication problem in real-time data using 
speech separation and speaker identification technique by Deep Learning and Image Processing 
algorithms. This will impact the communication and security sectors in a greater extent. Overall, this 
project aims to develop an application or method that can help to separate the audio-visual speech and 
enhance it based on speaker identification.
\\
This project can be further developed as:
\begin{itemize}
    \item By incorporating more real-world testing and gathering feedback from individual units.
    \item  The system can be connected with communication devices or services to enable the users to communicate with others with ease.
\end{itemize}


\cite{Tea} This project has a great potential to make a positive impact on communication and security situations. Its continuous improvement will be important to make this impact 
even greater example for citing and bibtex for journal paper \cite{croitoru2023diffusion}, conference paper \cite{mohamad2015smart}, citing website \cite{knuthwebsite}, citing book \cite{dirac}








\newpage

\pagestyle{plain}
\renewcommand{\bibname}{References}

\addcontentsline{toc}{chapter}{References}

\printbibliography



\end{document}
