\documentclass[12pt,a4paper]{report}
\usepackage[utf8]{inputenc}
\usepackage{amsfonts}
\usepackage{setspace}
\usepackage{graphicx}
\usepackage{array}
\usepackage{fancyhdr}
\usepackage{geometry}
\usepackage{ragged2e}
\usepackage{color}
\usepackage{biblatex}
\usepackage{tabularx}
\usepackage{listings}
\usepackage{xcolor} 
\usepackage{enumitem}
\usepackage{enumitem}


\lstdefinelanguage{JavaScript}{
  keywords={break, case, catch, class, const, continue, debugger, default, delete, do, else, export, extends, false, finally, for, function, if, import, in, instanceof, let, new, null, return, super, switch, this, throw, true, try, typeof, var, void, while, with, yield},
  keywordstyle=\color{blue},
  stringstyle=\color{red},
  commentstyle=\color{green},
  morecomment=[l][\color{magenta}]{//},
  morecomment=[s][\color{magenta}]{/*}{*/},
  morestring = [b]',
  morestring=[b]"
}

\lstset{
  backgroundcolor=\color{lightgray},
  basicstyle=\ttfamily\small,
  keywordstyle=\color{blue},
  commentstyle=\color{green},
  stringstyle=\color{red},
  numbers=left,
  numberstyle=\tiny\color{gray},
  stepnumber=1,
  numbersep=5pt,
  showspaces=false,
  showstringspaces=false,
  showtabs=false,
  frame=single,
  tabsize=2,
  captionpos=b,
  language=JavaScript,  % Set default language to JavaScript
}


\addbibresource{reference.bib}

\geometry{
a4paper,
total={210mm,297mm},
left=1.15in,
right=0.85in,
top=1.0in,
bottom=1.0in,
}

\begin{document}

\pagestyle{empty}

%%%%%%%%%%%%%%%%%%% Front Page  %%%%%%%%%%%%%%%%%%%%%%%
\begin{center}
{\large \textbf{Visvesvaraya Technological University, Belagavi – 590018}}
\begin{figure}[hbtp]
\centering
\includegraphics[width=2.3cm,height=3cm]{./pic/vtu.png}
\end{figure}

\textbf{MINI-PROJECT REPORT}
\par
\textbf{ON}
\par
\vspace{6pt}
{\Large \textbf{B-LOCK: Blockchain based eVault application}}
\par
\vspace{12pt}
\par
\textit{\textbf{Submitted in partial fulfillment for the award of degree of}}
\par
\vspace{12pt}
\large \textbf{BACHELOR OF ENGINEERING }
\par
\textbf{in}
\par
\large \textbf{COMPUTER SCIENCE \& ENGINEERING}
\par
\vspace{12pt}
\textit{\textbf{Submitted by}}

\begin{center}
\begin{tabular}{l@{\hspace{2cm}}r}
\textbf{\large Abhik L Salian } & \textbf{4SO21CS004} \\
\textbf{\large Ashwitha Shetty} & \textbf{4SO21CS030} \\
\textbf{\large H Karthik P Nayak } & \textbf{4SO21CS058} \\
\textbf{\large Jessica Lillian Mathew } & \textbf{4SO21CS066} \\
\end{tabular}
\end{center}

\vspace{12pt}
\textit{\textbf{Under the Guidance of}}
\par
\vspace{6pt}
\textbf{Dr. Santhosh Kumar DK}
\par
\vspace{2pt}
\normalsize {Associate Professor, Department of CSE}
\par
\begin{figure}[hbtp]
\centering
\includegraphics[scale=0.6]{./pic/sjeclogo.png}
\end{figure}
\large \textbf{DEPT. OF COMPUTER SCIENCE AND ENGINEERING}
\par \Large \textbf{ST JOSEPH ENGINEERING COLLEGE}
\par 
\textbf{An Autonomous Institution}
\par
{\large{(Affiliated to VTU Belagavi, Recognized by AICTE, Accredited by NBA)}}
\par
{\large \textbf{Vamanjoor, Mangaluru - 575028, Karnataka}}
\par 
{\Large \textbf{2023-24}}
\end{center}
\newpage

%%%%%%%%%%%%%%%%%%%%%%%%%% Certificate Page %%%%%%%%%%%%%%%%%%%%%%%%%%%%%%
\begin{center}
\LARGE \textbf{ST JOSEPH ENGINEERING COLLEGE}
\par
\Large \textbf{An Autonomous Institution}
\par \large{(Affiliated to VTU Belagavi, Recognized by AICTE, Accredited by NBA)}
\par \vspace{3pt}
\large \textbf{Vamanjoor, Mangaluru - 575028, Karnataka}
\par \vspace{12pt}  
\par
\large \textbf{DEPT. OF COMPUTER SCIENCE AND ENGINEERING}
\par
\begin{figure}[hbtp]
\centering
\includegraphics[scale=0.5]{./pic/sjeclogo.png}
\end{figure}


{\Large \textbf{CERTIFICATE}}
\end{center}
\justifying
\par
\setstretch{1.2}
\vspace{0.10in}
\noindent 
Certified that the Mini-project work entitled \textbf{``B-LOCK: Blockchain based eVault application''} carried out by\vspace{2pt} 
\par
\noindent 
\begin{center}
\begin{tabular}{l@{\hspace{2cm}}r}
\textbf{\large Abhik L Salian } & \textbf{4SO21CS004} \\
\textbf{\large Ashwitha Shetty} & \textbf{4SO21CS030} \\
\textbf{\large H Karthik P Nayak } & \textbf{4SO21CS058} \\
\textbf{\large Jessica Lillian Mathew } & \textbf{4SO21CS066} \\
\end{tabular}
\end{center}
\noindent
the bonafide students of VI semester Computer Science \& Engineering in partial fulfillment for the award of Bachelor of Engineering in Computer Science and Engineering of the Visvesvaraya Technological University, Belagavi during the year 2023-2024. It is certified that all suggestions indicated during internal assessment have been incorporated in the report. The project report has been approved as it satisfies the academic requirements in respect of project work prescribed for the said degree. 


\vspace{0.55in}
\par
\vspace{0.65in}
\setstretch{1.15}

\begin{tabularx}{0.95 \textwidth} { 
   >{\raggedright\arraybackslash}X 
   >{\centering\arraybackslash}X 
   >{\raggedleft\arraybackslash}X  }
     \textbf{Ms. L Hema} &  \textbf{Dr Sridevi Saralaya}\\
     Mini-Project Coordinator &   HOD-CSE \\
\end{tabularx}




%%%%%%%%%%%%%%%%%%%%%%%%%% Abstract %%%%%%%%%%%%%%%

\pagestyle{plain}
\setstretch{1.5}
\pagenumbering{roman}
\chapter*{Abstract}
\addcontentsline{toc}{chapter}{\numberline{}Abstract}

A blockchain-based eVault system is designed to enhance the 
security and efficiency of document management, addressing 
the growing challenge of safeguarding digital data. Traditional 
storage solutions often fall short in providing the necessary 
security and transparency for sensitive documents. In response, 
the eVault utilizes blockchain technology's decentralized and 
immutable characteristics to offer a robust platform for file 
storage, retrieval, and management. By encrypting documents and 
storing them on a distributed ledger, each file is assigned a 
unique identifier for efficient retrieval, while smart contracts 
enforce strict access control, ensuring that only authorized 
users can access stored data.

Testing and development indicate that the blockchain-based eVault 
significantly improves data integrity and reduces the risk of 
unauthorized access compared to conventional storage methods. 
Performance metrics demonstrate that the eVault maintains optimal 
transaction speeds and scalability, making it suitable for various 
industries. As a result, the eVault represents a significant 
advancement in document management, paving the way for improved 
digital security and privacy.

\setstretch{1.2}
\renewcommand{\contentsname}{Table of Contents}
\tableofcontents
\addcontentsline{toc}{chapter}{\numberline{}Table of Contents}
\listoffigures
\addcontentsline{toc}{chapter}{\numberline{}List of Figures}
\listoftables
%\addcontentsline{toc}{chapter}{\numberline{}List of Tables}
\newpage

%%%%%%%%%%%%%%%%%%%%% Headers and Footers %%%%%%%%%%%%%%%

\pagestyle{fancy}
\fancyhf{}
\lhead{\fontsize{10}{12} \selectfont B-LOCK: Blockchain based eVault application}
\rhead{\fontsize{10}{12} \selectfont Chapter \thechapter}
\lfoot{\fontsize{10}{12} \selectfont Department of Computer Science and Engineering, SJEC, Mangaluru}
\rfoot{\fontsize{10}{12} \selectfont Page \thepage}
\renewcommand{\headrulewidth}{0.5pt}
\renewcommand{\footrulewidth}{0.5pt}


%%%%%%%%%%%%%%%%%%%%%%% CHapetr 1 Introduction %%%%%%%%%%%%%

\setstretch{1.2}
\pagenumbering{arabic}

\chapter{Introduction}

\section{Background}

The blockchain-based eVault project aims to revolutionize document 
management by utilizing blockchain technology to significantly 
enhance security and efficiency. In an era where digital data is 
rapidly expanding, traditional storage solutions often lack the 
robust protection needed to prevent breaches and unauthorized 
access. This project addresses these vulnerabilities by developing 
a decentralized system that leverages blockchain’s immutable 
ledger to safeguard sensitive information.

Conventional storage methods frequently fail to meet modern 
security demands, making blockchain’s decentralized and 
tamper-proof characteristics a compelling solution. By 
incorporating these features, the eVault system offers a 
secure and efficient way to manage documents, addressing 
critical issues in digital privacy and data integrity.


\section{Problem statement }
Traditional document storage systems often fail to ensure 
adequate data security and integrity, making them vulnerable 
to breaches, unauthorized access, and tampering. This problem 
is made worse by the growing need for secure and transparent 
storage solutions.

Organizations managing sensitive or regulatory data face severe 
risks, including data loss and compliance violations, due to 
these weaknesses. To address these challenges, a blockchain-based 
eVault system is proposed to offer a secure, decentralized 
solution for document management, enhancing security, transparency, 
and data integrity.
\newpage
\section{Scope}
The project aims to develop a blockchain-based eVault system 
designed to enhance the security and efficiency of document 
management. Its primary goals include providing a secure, 
decentralized platform for storing, retrieving, and managing 
sensitive documents. By leveraging blockchain technology, the 
project seeks to ensure data integrity, transparency, and access 
control through encryption and smart contracts.

Achievements include the successful development of a fully 
functional eVault application that securely encrypts and stores 
files on a blockchain. The system features real-time document 
retrieval, user authentication, and role-based access control, 
demonstrating robust performance and scalability. The project has 
also effectively addressed the limitations of traditional document 
storage systems, offering improved security and transparency.
%---------------------------- Chapter TWO --------------------------

\chapter{Software Requirements Specification}

\section{Introduction}

This Software Requirements Specification (SRS) document outlines 
the requirements for the blockchain-based eVault system, aimed at 
improving document management security and efficiency. The system 
is designed to offer a secure, decentralized platform for storing, 
retrieving, and managing sensitive documents using blockchain 
technology, ensuring data integrity and transparency through 
encryption and smart contracts \cite{antonopoulos2018mastering}. 
The eVault aims to provide a secure, decentralized platform for 
storing, retrieving, and managing documents, addressing the 
limitations of traditional storage systems \cite{nodedocs, reactdocs,firebasedocs}. 
It details the functional and non-functional requirements, user 
interface design, and performance criteria for the project. This 
document is intended for use by developers and stakeholders to 
ensure the system aligns with its objectives and constraints. 
Key resources for this project include official documentation 
and best practices for Node.js \cite{nodedocs}, React \cite{reactdocs}, 
Firebase \cite{firebasedocs}, and related technologies.

\section{Functional Requirements}

Functional requirements define the system’s expected behaviors and interactions. They outline the essential functions and features that the system must support.
\begin{enumerate}

\item \textbf{ User Interactions:}
    \begin{itemize}
        \item \textbf{Login/Logout:} Secure authentication and user session management.
        \item \textbf{File Management:} Features for uploading, downloading, and deleting files.
        \item \textbf{Access Control:} Permissions and role-based access to restrict or grant file access.
    \end{itemize}
\item \textbf{System Operations:}
    \begin{itemize}
        \item \textbf{Data Retrieval:} Efficient file retrieval using unique identifiers.
        \item \textbf{Document Encryption:} Encryption methods to ensure data confidentiality.
    \end{itemize}
\item \textbf{Data Management}:
    \begin{itemize}
        \item \textbf{Data Encryption}: Procedures for encrypting sensitive data both in transit and at rest.
        \item \textbf{Indexing}: Methods for indexing and organizing files for efficient access.
        \item \textbf{Storage Formats}: Supported file formats and storage requirements.
    \end{itemize}
\item \textbf{Error Handling}:
    \begin{itemize}
        \item \textbf{Upload Failures}: Mechanisms for handling failed file uploads.
        \item \textbf{Access Issues}: Procedures for managing invalid access attempts or permission errors.
    \end{itemize}
\item \textbf{Integration Points}:
    \begin{itemize}
        \item \textbf{Authentication Services}: Integration with external authentication services.
        \item \textbf{External Databases}: Connectivity and interaction with other databases or services.
    \end{itemize}
    
\end{enumerate}

\section{Non-Functional Requirements}

Non-functional requirements describe the quality attributes, performance standards, and other system constraints that are not related to specific functionalities but are crucial for the system’s overall effectiveness and user satisfaction.

\begin{enumerate}
    \item \textbf{Performance:}
        The eVault system must support a response time of less than 2 seconds for file retrieval requests.
    
    \item \textbf{Scalability:}
        The system must be able to scale horizontally to accommodate additional users and data storage requirements.
    
    \item \textbf{Security:}
    User access must be controlled via multi-factor authentication and role-based permissions.
    
    \item \textbf{Reliability:}
        The eVault must achieve 99.9\% uptime, ensuring high availability.
        Backup procedures must be in place, with daily backups stored in a secure, offsite location.
    
    \item \textbf{Compatibility:}
        The eVault must be compatible with all major web browsers, including Chrome, Firefox, and Edge.


\end{enumerate}

   \section{User Interface Requirements}

   This section outlines the design and functional requirements of the user interface (UI) for the eVault system. It includes considerations for usability, accessibility, and aesthetic appeal, ensuring an intuitive experience for all users.
   
   \subsection{Design Principles}
   The user interface should adhere to modern design principles to enhance usability and user satisfaction.
   \begin{itemize}
       \item \textbf{Consistency:} The UI should maintain consistency in design elements across all screens, including fonts, colors, and button styles.
       \item \textbf{Simplicity:} The design should be clean and clutter-free, emphasizing essential functions to avoid overwhelming the user.
       \item \textbf{Clarity:} The interface should be organized in a way that makes it easy for users to navigate and find features. Key actions such as file upload, download, and account management should be easily accessible.
   \end{itemize}
   
   \subsection{Accessibility}
   The application should be accessible to all users, including those with disabilities, by adhering to accessibility standards.
   \begin{itemize}
        \item \textbf{Consistent Navigation:} Design the application's navigation to be consistent across all pages, allowing users to predictably move through the application without confusion.
        \item \textbf{Flexible Input Methods:} Support multiple input methods, such as touch, keyboard, and mouse, to cater to different user preferences and abilities.
        \item \textbf{Color Contrast:} Ensure sufficient contrast between text and background colors to improve readability for users with visual impairments.
   \end{itemize}
   
   \subsection{User Feedback}
   Feedback mechanisms should be implemented to gather user input and improve the application over time.
   \begin{itemize}
       \item \textbf{Error Messages:} Provide clear and informative error messages to guide users in resolving issues.
       \item \textbf{User Surveys:} Periodic surveys can be conducted to collect user feedback on the UI and identify areas for improvement.
   \end{itemize}
   
   \section{Software Requirements}

This section outlines the technical requirements for the eVault system, addressing hardware, software, network, and operational needs to ensure efficient and reliable performance.

\subsection{Hardware Requirements}
The eVault system requires specific hardware configurations to deliver optimal performance and support user needs.
\begin{itemize}
    \item \textbf{Server Specifications:} Requires a minimum of 16 GB RAM, a 4-core CPU, and a 500 GB SSD.
    \item \textbf{User Devices:} Devices must have at least 4 GB RAM and a modern web browser for optimal functionality.
\end{itemize}

\subsection{Software Dependencies}
The system relies on several software components and libraries for its functionality.
\begin{itemize}
    \item \textbf{Operating Systems:} Compatible with Windows Server 2019 and macOS Mojave (10.14).
    \item \textbf{Database Systems:} Utilizes Node.js and Firebase for data management.
    \item \textbf{Tech Stack:} Developed with HTML, CSS, JavaScript, and Solidity, utilizing the React library for building user interfaces.
\end{itemize}






\chapter{System Design}
\section{Architecture Design}
\begin{figure}[hbtp]
\centering
\includegraphics[width=5.5in,height=3in]{./pic/architecture.drawio.png}
\caption{System Architecture Diagram}
\label{fig:architecture}
\end{figure}
Figure \ref{fig:architecture} illustrates the high-level architecture of the B-LOCK system. This architecture comprises several key components: the user interface, blockchain smart contracts, and the storage backend. The user interface allows users to interact with the application, facilitating file uploads and retrieval. The blockchain smart contracts, deployed on Ethereum, handle file metadata storage and verification. The storage backend, implemented using Firebase, manages the actual file storage and retrieval. This design ensures a secure and user-friendly application where file integrity and ownership are maintained through blockchain technology.
\section{Decomposition Description}
\begin{figure}[hbtp]
\centering
\includegraphics[width=1.7in,height=4in]{./pic/flowchart.png}
\caption{Flow chart}
\label{fig:flowchart}
\end{figure}
Figure \ref{fig:flowchart} presents a decomposition diagram (or a flow chart) of the B-LOCK system. This diagram breaks down the system into its core modules: user authentication, file upload, blockchain integration, and data retrieval. Each module is responsible for specific functionalities:
\begin{itemize}
   \item \textbf{User Authentication:} Manages user sign-in and sign-up, including email verification and session management.
   \item \textbf{File Upload:} Handles file selection, uploading to Firebase Storage, and associating file metadata with the blockchain.
   \item \textbf{Blockchain Integration:} Interacts with Ethereum smart contracts to store and retrieve file hashes and metadata.
   \item \textbf{Data Retrieval:} Provides mechanisms for users to access their uploaded files and associated data.
\end{itemize}


\section{Data Flow Design}

\begin{figure}[hbtp]
\centering
\includegraphics[scale=0.7]{./pic/dataflow.png}
\caption{Dataflow Diagram}
\label{fig:dataflow}
\end{figure}
Figure \ref{fig:dataflow} depicts the data flow within the B-LOCK application. The diagram illustrates the process from file upload to storage and retrieval. When a user uploads a file, the file is first hashed and then uploaded to Firebase Storage. The file's metadata, including its name and hash, is sent to the Ethereum smart contract for secure storage on the blockchain. Upon successful upload, the application updates the user interface to reflect the new file and its status. During file retrieval, the application queries both Firebase and the blockchain to provide users with the necessary information and access to their files.



\chapter{Implementation}
\section{System Overview}
\par
The system is a blockchain-based e-vault application designed to securely store and manage files with metadata. The core components include:
\begin{itemize}
   \item \textbf{Frontend:} A React application\cite{reactdocs} for user interaction created using Node Package Manager(npm)\cite{npmdocs}.
   \item \textbf{Backend:} Firebase\cite{firebasedocs} for file storage and Firestore for metadata management.
   \item \textbf{Blockchain:} Ethereum for file hash storage and verification.
\end{itemize}

\section{Pseudocode and Algorithms}

\subsection{File Upload Algorithm}

\begin{lstlisting}[caption=File Upload Algorithm]
   function uploadFile(file):
    hash = generateFileHash(file.name, currentTimestamp)
    storageRef = createStorageReference(user.uid, hash, file.name)
    uploadTask = uploadFileToStorage(storageRef, file)

    onUploadProgress(uploadTask):
        progress = calculateUploadProgress(uploadTask)
        displayUploadProgress(progress)

    onUploadComplete(uploadTask):
        downloadURL = getDownloadURL(uploadTask)
        storeFileHashInBlockchain(fileName, fileHash)
        saveFileMetadataInFirestore(fileName, downloadURL, 
        user.uid)
        displaySuccessMessage("File uploaded successfully")

    onUploadError(uploadTask):
        displayErrorMessage("Error uploading file")

   \end{lstlisting}
   
\begin{itemize}
   \item \textbf{Generate File Hash:} Create a unique hash using the file name and current timestamp.
   \item \textbf{Create Storage Reference:} Create a reference in Firebase Storage using the hash and file name.
   \item \textbf{Upload File:} Perform the file upload asynchronously.
   \item \textbf{Progress Monitoring:} Update progress based on the upload state.
   \item \textbf{Completion Handling:} On success, store the file hash in the blockchain and metadata in Firestore.
   \item \textbf{Error Handling:} Display appropriate error messages if the upload fails.
\end{itemize}

\subsection{User Authentication Algorithm}
\begin{lstlisting}[caption=User Authentication Algorithm]
   function signIn(email, password):
    try:
        user = authenticateUserWithEmail(email, password)
        if user.isEmailVerified:
            redirectToDashboard()
        else:
            displayErrorMessage("Email not verified")
    except AuthenticationError as e:
        displayErrorMessage(e.message)
   \end{lstlisting}
\begin{itemize}
   \item \textbf{Authenticate User:} Check user credentials against Firebase Authentication.
   \item \textbf{Email Verification Check:} Ensure the user’s email is verified before granting access.
   \item \textbf{Error Handling:} Handle and display errors during authentication.
\end{itemize}


\section{Key Components Implementation}
\subsection{Smart Contract}
The EVault smart contract\cite{antonopoulos2018mastering}, written in Solidity and compiled and migrated using Truffle\cite{truffledocs}, is integral to the B-LOCK e-vault application. It stores file names and hashes on the Ethereum blockchain to ensure file integrity and security. The contract has three main functions: storeFile for recording file details, retrieveFiles for fetching a user's files, and getFilenames for listing file names. This ensures tamper-proof, decentralized storage and secure file management, leveraging blockchain's transparency and immutability.
\begin{lstlisting}[caption=Solidity Smart Contract (EVault.sol)]
    pragma solidity ^0.8.0;

contract EVault {
    struct File {
        string fileName;
        string fileHash;
    }

    mapping(address => File[]) private userFiles;

    function storeFile(string memory _fileName, 
    string memory _fileHash) public {
        userFiles[msg.sender].push(File(_fileName, _fileHash));
    }

    function retrieveFiles() public view returns (File[] memory){
        return userFiles[msg.sender];
    }
    
    function getFilenames() public view returns (string[] memory){
        File[] memory files = userFiles[msg.sender];
        string[] memory fileNames = new string[](files.length);

        for (uint i = 0; i < files.length; i++) {
            fileNames[i] = files[i].fileName;
        }

        return fileNames;
    }
}
    \end{lstlisting}
\subsection{Frontend (React)}
\par
The frontend is developed using React.js and consists of the following key components:
\begin{itemize}
   \item \textbf{SignIn Component:} Handles user login.
   \item \textbf{SignUp Component:} Manages user registration and email verification.
   \item \textbf{Upload Component:} Manages file upload and progress display.
   \item \textbf{Retrieve Component:} Allows users to view and manage uploaded files.
\end{itemize}
\begin{lstlisting}[caption=Frontend Code Snippet]
   import React, { useState } from 'react';
   import { signInWithEmailAndPassword } from 'firebase/auth';
   import { auth } from './firebase';
   
   const SignIn = ({ onSignIn, switchToSignUp }) => {
       const [email, setEmail] = useState('');
       const [password, setPassword] = useState('');
       const [error, setError] = useState('');
   
       const handleSubmit = async (e) => {
           e.preventDefault();
           try {
               const userCredential = await 
               signInWithEmailAndPassword(auth, email, password);
               const user = userCredential.user;
   
               if (user.emailVerified) {
                   onSignIn(email, password);
                   setError(''); // Clear previous errors
               } else {
                   setError('Please verify your email before 
                   signing in.');
                   alert('Email not verified. Please check your 
                   inbox for the verification email.');
               }
           } catch (err) {
               console.error("Sign-in error:", err);
               setError(err.message);
           }
       };
   
       return (
           <div className="auth-form">
               <h2>Sign In</h2>
               <form onSubmit={handleSubmit}>
                   <input 
                       type="email" 
                       placeholder="Email" 
                       value={email} 
                       onChange={(e) => setEmail(e.target.value)} 
                       required 
                   />
                   <input 
                       type="password" 
                       placeholder="Password" 
                       value={password} 
                       onChange={(e) => setPassword
                       (e.target.value)} 
                       required 
                   />
                   <button type="submit">Sign In</button>
               </form>
               {error && <p className="error">{error}</p>}
               <p>Don't have an account? <button 
               onClick={switchToSignUp}>Sign Up</button></p>
           </div>
       );
   };
   
   export default SignIn;
   
   \end{lstlisting}

\subsection{Backend (Firebase and Blockchain)}
\begin{itemize}
   \item \textbf{Firebase Storage:} Stores the uploaded files.
   \item \textbf{Firestore Database:} Saves metadata about the files.
   \item \textbf{Ethereum Blockchain:} Stores the file hashes for verification.
\end{itemize}
\begin{lstlisting}[caption=Backend Code Snippet]
import { ref, uploadBytesResumable, getDownloadURL } from 
"firebase/storage";
import { collection, addDoc } from "firebase/firestore";

const uploadFile = async (file, user, storage, db, evault, 
account) => {
    const hash = CryptoJS.SHA256(file.name + new Date().
    toISOString()).toString();
    const storageRef = ref(storage, `uploads/${user.uid}/${hash}
    _${file.name}`);
    const uploadTask = uploadBytesResumable(storageRef, file);

    uploadTask.on(
        "state_changed",
        (snapshot) => {
            const progress = (snapshot.bytesTransferred / 
            snapshot.totalBytes) * 100;
            console.log(`Upload is ${progress.toFixed(2)}% done`);
        },
        (error) => {
            console.error("Upload error:", error);
        },
        async () => {
            const downloadURL = await getDownloadURL(uploadTask.
            snapshot.ref);
            try {
                await evault.methods.storeFile(file.name, hash)
                .send({ from: account });
                await addDoc(collection(db, "files"), {
                    name: file.name,
                    owner: user.uid,
                    downloadURL,
                    timestamp: new Date(),
                });
                console.log("File uploaded successfully");
            } catch (error) {
                console.error("Error saving file data", error);
            }
        }
    );
};

   \end{lstlisting}


\section{Integration}
\par The frontend components interact with the Firebase 
backend and Ethereum blockchain to provide a seamless user 
experience. Authentication is handled via Firebase 
Authentication, ensuring that users are securely signed in. 
File uploads are managed through Firebase Storage, allowing 
users to upload and access their files easily. Metadata related 
to the files, such as names and download URLs, is stored in 
Firestore, providing quick and reliable access to file 
information. The integration with the Ethereum blockchain 
guarantees that file hashes are securely stored and verifiable, 
leveraging smart contracts to maintain the integrity and 
ownership of each file. This combination of Firebase and 
blockchain technologies ensures that the application is both 
secure and scalable, providing a robust solution for digital
 file management.




\chapter{Results and Discussion}
\section{Results}
\par After successfully implementing B-LOCK, our blockchain-based e-vault application, we conducted a series of tests to evaluate its performance and functionality. Below are the key results from these tests:
\begin{enumerate}
   \item User Authentication and Verification:
   \begin{itemize}
      \item \textbf{Result:} The application successfully authenticated users via Firebase Authentication.
      \item \textbf{Discussion:} The integration of Firebase ensured a smooth and secure sign-in process. Users received verification emails promptly, and only verified users could upload and access files.
   \end{itemize}
   \item File Upload and Storage:
   \begin{itemize}
      \item \textbf{Result:} Files were uploaded to Firebase Storage and their metadata was correctly stored in Firestore.
      \item \textbf{Discussion:} Upload speeds were satisfactory, and users were able to retrieve download URLs without delay. The metadata storage in Firestore allowed for efficient file management and retrieval.
   \end{itemize}
   \item Blockchain Integration:
   \begin{itemize}
      \item \textbf{Result:} File hashes were successfully stored on the Ethereum blockchain.
      \item \textbf{Discussion:} Storing file hashes on the blockchain provided an additional layer of security, ensuring the integrity and ownership of files. Users could verify their file's existence and authenticity using blockchain transactions.
   \end{itemize}
   \item Performance Metrics:
   \begin{itemize}
      \item \textbf{Result:} The system handled concurrent uploads and downloads efficiently.
      \item \textbf{Discussion:} The application's performance remained stable under various load conditions. The integration of Firebase and blockchain technologies did not significantly impact the responsiveness of the application.
   \end{itemize}
\end{enumerate}

\section{Discussion}
The development of B-LOCK has shown promising results in creating a secure and reliable e-vault application using blockchain technology. Several key aspects were highlighted during the development and testing phases:
\begin{itemize}
   \item \textbf{Security:} The use of Firebase Authentication provided a robust mechanism for user verification. Combining this with blockchain for storing file hashes enhanced security by ensuring file integrity and ownership, making it difficult for unauthorized users to tamper with the stored data.
   \item \textbf{Scalability:} The architecture of B-LOCK allows for scalability. Firebase’s cloud-based solutions can handle an increasing number of users and files. The use of blockchain, while slightly increasing transaction time due to network confirmations, scales effectively with user demand for verification and integrity checks.
   \item \textbf{User Experience:} User feedback indicated a positive experience with the interface and the ease of use for both uploading and retrieving files. The clear instructions and responsive design contributed to a smooth user journey.
   \item \textbf{Integration Challenges:} Integrating blockchain with traditional cloud services like Firebase posed some challenges, particularly in ensuring seamless interaction between different technologies. However, these were overcome with thorough testing and incremental integration approaches.
\end{itemize}

\section{Screenshots}
Below are screenshots of the B-LOCK application demonstrating key features and user interface elements.

\subsection{User Registration and Login}
\begin{figure}[hbtp]
    \centering
    \includegraphics[scale=0.4]{./pic/signup.png}
    \caption{Sign Up Page to register new users}
    \label{fig:signup}
\end{figure}
\begin{figure}[hbtp]
    \centering
    \includegraphics[scale=0.4]{./pic/signin.png}
    \caption{Sign In Page to login already registered users}
    \label{fig:signin}
\end{figure}
\begin{figure}[hbtp]
    \centering
    \includegraphics[scale=0.4]{./pic/emailver.png}
    \caption{Verification email that users receive after signing up}
    \label{fig:emailver}
\end{figure}
Figures \ref{fig:signup} and \ref{fig:signin} show the user registration and login interface. Users can create an account by entering their email and password, and existing users can sign in using their credentials. Upon successful registration, a verification email is sent to the user to confirm their email address before accessing the application, as shown in Figure \ref{fig:emailver}.

\subsection{File Upload Interface}
\begin{figure}[hbtp]
    \centering
    \includegraphics[scale=0.4]{./pic/fileupload.png}
    \caption{File Upload Interface}
    \label{fig:fileupload}
\end{figure}
The file upload interface, as shown in Figure \ref{fig:fileupload}, allows users to select files from their local device to upload to the e-vault. The interface supports multiple file uploads and provides feedback on the upload progress. Users are notified of the upload status, ensuring they are aware of successful uploads or any errors encountered during the process.

\subsection{File Retrieve Interface}
\begin{figure}[hbtp]
    \centering
    \includegraphics[scale=0.4]{./pic/retrieve.png}
    \caption{File Retrieve Interface that displays a list of uploaded files}
    \label{fig:retrieve}
\end{figure}
As shown in Figure \ref{fig:retrieve}, the file retrieval interface allows users to access the files they have previously uploaded to the e-vault. This interface provides a list of files with details such as file name, upload date, and a download link. Users can easily search for specific files, sort the list based on different criteria, and download files directly to their devices. Additionally, the interface shows the verification status of each file, indicating whether the file hash has been successfully stored on the blockchain, thus ensuring the integrity and authenticity of the files.
\subsection{File Details Getting Stored in Blockchain}
\begin{figure}[hbtp]
    \centering
    \includegraphics[scale=0.4]{./pic/ganache.png}
    \caption{File name and file hash getting stored in blocks in Ganache}
    \label{fig:ganache}
\end{figure}
Figure \ref{fig:ganache} shows the file details details getting stored as blocks in Ganache\cite{ganachedocs} on Ethereum\cite{ethereumdocs} local network. After successful deduction of ethers to upload the file, the file details gets stored here.


\chapter{Conclusion and Future Work}
\section{Conclusion}
% \cite{bashir2021subjective} \cite{mittal2016}
The B-LOCK project successfully demonstrates the integration of blockchain technology with file storage solutions to create a secure e-vault application. Throughout the development process, we addressed critical aspects of data security, user authentication, and file integrity. By leveraging Ethereum's blockchain, we ensured that every uploaded file is securely hashed and its integrity verifiable, thus preventing unauthorized tampering or data breaches.
\\
Key accomplishments of the B-LOCK project include:
\begin{itemize}
\item \textbf{Secure File Uploads:} Implementation of an intuitive interface allowing users to upload files securely.
\item \textbf{Blockchain Verification:} Integration with the Ethereum blockchain to store file hashes, ensuring the authenticity and integrity of uploaded files.
\item \textbf{User Authentication:} A robust user registration and login system, including email verification for added security.
\item \textbf{Efficient File Management:} A user-friendly dashboard for managing uploaded files, with options to view, download, and delete files.

\end{itemize}

\section{Future Work}
While the current implementation of B-LOCK meets its primary objectives, there are several areas for potential improvement and expansion. Future work could focus on the following aspects:
\begin{itemize}
   \item \textbf{Enhanced Scalability:} As the user base grows, optimizing the system for better scalability will be crucial. This may involve exploring more efficient blockchain solutions or layer-2 scaling techniques.
   \item \textbf{Mobile Application:} Developing a mobile version of B-LOCK to extend accessibility and usability across various devices.
   \item \textbf{Advanced Security Features:} Integrating additional security measures such as multi-factor authentication (MFA), biometric verification, and encryption for stored files.
   \item \textbf{Decentralized Storage Integration:} Exploring decentralized storage solutions such as IPFS (InterPlanetary File System) or Filecoin to further enhance data availability and redundancy.
   \item \textbf{Smart Contract Enhancements:} Improving the smart contract logic to include features like file versioning, access control, and more sophisticated file-sharing mechanisms.
   \item \textbf{User Experience Improvements:} Continuously refining the user interface and experience based on user feedback and emerging best practices in UI/UX design.
   \item \textbf{Automated Backup and Recovery:} Implementing automated backup and disaster recovery solutions to ensure data availability and integrity in case of system failures.
\end{itemize}
By addressing these areas, B-LOCK can evolve into a more comprehensive and versatile e-vault solution, offering enhanced security, scalability, and user experience. The project's success serves as a testament to the potential of blockchain technology in revolutionizing secure data storage and management systems.


% \cite{Tea} This project has a great potential to make a positive impact on communication and security situations. Its continuous improvement will be important to make this impact 
% even greater example for citing and bibtex for journal paper \cite{croitoru2023diffusion}, conference paper \cite{mohamad2015smart}, citing website \cite{knuthwebsite}, citing book \cite{dirac}




\newpage

\pagestyle{plain}
\renewcommand{\bibname}{References}

\addcontentsline{toc}{chapter}{References}

\printbibliography



\end{document}

